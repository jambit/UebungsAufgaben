\documentclass[12pt,a4paper]{article}
\usepackage[a4paper,includeheadfoot,margin=2.14cm,footskip=40pt,bottom=0.7cm]{geometry}
% ------------------------- packages ------------------------------------------
% encoding and language
\usepackage[utf8]{inputenc}
\usepackage[english,ngerman]{babel}

% Design
\usepackage{graphicx}
\usepackage{background}
\usepackage{fancyhdr}
\usepackage{lastpage}

\usepackage[ddmmyyyy]{datetime}
\renewcommand{\dateseparator}{.}

\usepackage{helvet}
\renewcommand{\familydefault}{\sfdefault}

% code
\usepackage{listings}

\graphicspath{{../graphics/}}

\backgroundsetup{scale = 1, angle = 0, opacity = 1,
   contents = {\includegraphics[width = \paperwidth,
   height = \paperheight, keepaspectratio, page=1]
   {jambit_watermark.pdf}}}

\pagestyle{fancy}
\fancyhf{} % clear all header and footer fields
% remove header
\renewcommand{\headrulewidth}{0pt}
\fancyhead{}
% style footer
\fancyfoot[L]{\today}
\fancyfoot[C]{\uppercase{Azubi Übungsaufgabe}}
\fancyfoot[R]{\thepage/\pageref{LastPage}}
\fancyfootoffset[L]{25pt}
\fancyfootoffset[R]{25pt}

\definecolor{orange}{HTML}{f07f3c}
\definecolor{grau}{HTML}{58585a}

\newcounter{aufgaben}
\setcounter{aufgaben}{0}

\setlength\parindent{0pt}

\newcommand{\titel}[1]{
  {
  \color{orange}\LARGE\textsf{\textbf{#1}}\\
  \vspace{-0.8cm}\rule{\textwidth}{1pt}
  }
}

\newcommand{\aufgabe}[1]{
  {
  \addtocounter{aufgaben}{1}
  \vspace{0.5cm}\\
  \large\color{grau}\textbf{\arabic{aufgaben}) #1}
  \vspace{0.2cm}\\
  }
}


\begin{document}
\titel{Java Basics}
Ziel dieses Aufgabenblocks \textsl{Java Basics} soll sein, dass Basiswissen zum Programmieren mit Java aufgebaut und gefestigt wird. Zur Lösung der Aufgaben ist es \underline{nicht} notwendig zusätzliche Libraries einzubinden. Es soll und darf ausschließlich mit Java Boardmitteln gearbeitet werden. Mindestens sollte \textbf{Java 11} verwendet werden.
\aufgabe{Taschenrechner}
Zum erlernen und festigen des Basiswissens soll im Folgenden ein zunächst simpler Taschenrechner implementiert werden. Dieser soll Benutzereingabe über die Komandozeile akzeptieren und verarbeiten können sowie die Ergebnisse auf dieser für einen Menschen lesebar ausgeben.
\unteraufgabe{Addition und Subtraktion}
Implementiere im ersten Schritt die Addition und Subkration von Ganzzahlen $\mathbb{Z}$ und Reellen Zahlen $\mathbb{R}$.\\
Ein- und Ausgaben könnte wie folgt aussehen, wobei alle möglichen Eingabevariante zu einem Ergebnis führen müssen.
\begin{codeblock}
15 - 6
Ergebnis: 9
1-3
Ergebnis: -2
10-3-3
Ergebnis: 4
21            -1      // viele Leerzeichen vor und hinter jedem Zeichen
Ergebnis: 20
1.8 - 0.3
Ergebnis: 1.5
\end{codeblock}
\unteraufgabe{Multiplikation und Division}
Analog zur vorherigen Aufgabe soll nun die Multiplikation und Division zusätzlich implementiert werden. Dabei gelten die selben Anforderung wie bei der Addition und Subtraktion.\\
Wenn du dein Programmcode bereits in der ersten Aufgabe gut strukuriert hast, ist dies ein Quickwin da jediglich die beiden zusätzlichen Rechenoperationen hinzugefügt werden müssen.
\unteraufgabe{Wurzel- und Potenzrechnung}
Erweitere deinen Taschenrechner um weitere Rechenoperatoren um Wurzel und Potenzrechnung durchführen zukönnen. Wähle dazu geeignete Operatorenzeichen für die Benutzereingabe.\\
Beispiele für mögliche Eingaben:
\begin{codeblock}
2^2
5 pow 3
8 sqrt 3
\end{codeblock}
\unteraufgabe{Bonus: Größere Gleichungen}
Implementiere die Möglichkeit längere Gleichungen eingeben zu können ohne dabei die Grundrechenregeln zu verletzen.
\begin{codeblock}
3+2*5
2*5+5*4
2*5/2
\end{codeblock}
\unteraufgabe{Bonus: Klammerung}
Implementiere die Möglichkeit explitzite Klammerung in einer Gleichung verwendet zu können. Wobei innerhalb und außerhalb eines Klammernblocks weiterhin die Grundrechenregeln nicht verletzt werden dürfen.
\begin{codeblock}
(3+2)*5
2*5+5*(4+2)
(2*5/2)+0.5*2
\end{codeblock}
\end{document}
