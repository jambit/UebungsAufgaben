\documentclass[12pt,a4paper]{article}
\usepackage[a4paper,includeheadfoot,margin=2.14cm,footskip=40pt,bottom=0.7cm]{geometry}
% ------------------------- packages ------------------------------------------
% encoding and language
\usepackage[utf8]{inputenc}
\usepackage[english,ngerman]{babel}

% Design
\usepackage{graphicx}
\usepackage{background}
\usepackage{fancyhdr}
\usepackage{lastpage}

\usepackage[ddmmyyyy]{datetime}
\renewcommand{\dateseparator}{.}

\usepackage{helvet}
\renewcommand{\familydefault}{\sfdefault}

% code
\usepackage{listings}

\graphicspath{{../graphics/}}

\backgroundsetup{scale = 1, angle = 0, opacity = 1,
   contents = {\includegraphics[width = \paperwidth,
   height = \paperheight, keepaspectratio, page=1]
   {jambit_watermark.pdf}}}

\pagestyle{fancy}
\fancyhf{} % clear all header and footer fields
% remove header
\renewcommand{\headrulewidth}{0pt}
\fancyhead{}
% style footer
\fancyfoot[L]{\today}
\fancyfoot[C]{\uppercase{Azubi Übungsaufgabe}}
\fancyfoot[R]{\thepage/\pageref{LastPage}}
\fancyfootoffset[L]{25pt}
\fancyfootoffset[R]{25pt}

\definecolor{orange}{HTML}{f07f3c}
\definecolor{grau}{HTML}{58585a}

\newcounter{aufgaben}
\setcounter{aufgaben}{0}

\setlength\parindent{0pt}

\newcommand{\titel}[1]{
  {
  \color{orange}\LARGE\textsf{\textbf{#1}}\\
  \vspace{-0.8cm}\rule{\textwidth}{1pt}
  }
}

\newcommand{\aufgabe}[1]{
  {
  \addtocounter{aufgaben}{1}
  \vspace{0.5cm}\\
  \large\color{grau}\textbf{\arabic{aufgaben}) #1}
  \vspace{0.2cm}\\
  }
}


\begin{document}
\titel{Java Basics}
\aufgabe{Taschenrechner}
Determine the largest k such that the table is k-anonym. Explain which rows contradict the (k+1)-
anonymity. The dataset is 2-anonymous, as there is no Quasi-Identi fi er-tuple which occurs only once.
It is not 3-anonymous, as e.g. (F; 24; 10000) occurs only twice.
2. You may now use suppression on the columns. Assume that by removing one digit from Age or Zip,
or suppressing the Sex att ribute, you lose one ”value”. What is the minimal value loss required to
achieve 5-anonymity?
5-anonymity can be achieved by suppressing the last digit of Age and the last digit of Zip. Hence,
the minimal value is at most 2. It is not 1 as:
Suppressing Sex leads to 2-anonymity, e.g. ( ; 24; 10000) occurs only twice.
Suppressing the last digit of Age leads to 2-anonymity, e.g. (F; 2 ; 10001) occurs only twice.
Suppressing the fi rst digit does not give any benefi t, as all age numbers begin with ”2”.
Suppressing the last digit of Zip leads to 2-anonymity, e.g. (F; 24; 1000 ) occurs only twice.
Suppressing any other digit does not give any benefi t, as all zip codes begin with ”1000”.
\aufgabe{Taschenrechner Informatik}
Disti nct l-Diversity
1. What is one shortcoming of k-anonymity compared to l-diversity? Which att ack exploits this weakness?
k-anonymity only regards the quasi-identi fi ers, but does not investi gate the distributi on of the sensiti ve
att ribute within one equivalence-class w.r.t. the quasi-identi fi er. This can be exploited by the
Background-Knowledge Att ack.
2. Given that a dataset is k-anonymous, but not (k +1)-anonymous. What implicati ons does this have
on the disti nct l-diversity of the dataset? Give a lower and upper bound for l.
The smallest equivalence-class w.r.t. to the Quasi-Identi fi er has size k. Hence, in this class there
can only be at most k diff erent values for the sensiti ve att ribute. Thus, l can be bounded from above
as l k. Trivially, 1 l holds as lower bound. As k-anonymity does not make any statement
about the distributi on of the sensiti ve att ribute, we cannot guarantee a larger lower bound, i.e. the
following bounds are ti ght: 1 l k.
3. Knowing only the distributi on of the sensiti ve att ribute values; What bounds can you derive for l in
disti nct l-diversity?
Let L be the number of diff erent sensiti ve att ribute values. Then, there can also be at most L diff erent
values within each equivalence class w.r.t. to an Quasi-Identi fi er. Thus, l L.
Additi onal informati on: This bound is independent of the bound from (ii), as the former one operates
only on the Quasi-Identi fi er columns and this one solely considers the sensiti ve att ribute.
Disti nct l-Diversity
1. What is one shortcoming of k-anonymity compared to l-diversity? Which att ack exploits this weakness?
k-anonymity only regards the quasi-identi fi ers, but does not investi gate the distributi on of the sensiti ve
att ribute within one equivalence-class w.r.t. the quasi-identi fi er. This can be exploited by the
Background-Knowledge Att ack.
2. Given that a dataset is k-anonymous, but not (k +1)-anonymous. What implicati ons does this have
on the disti nct l-diversity of the dataset? Give a lower and upper bound for l.
The smallest equivalence-class w.r.t. to the Quasi-Identi fi er has size k. Hence, in this class there
can only be at most k diff erent values for the sensiti ve att ribute. Thus, l can be bounded from above
as l k. Trivially, 1 l holds as lower bound. As k-anonymity does not make any statement
about the distributi on of the sensiti ve att ribute, we cannot guarantee a larger lower bound, i.e. the
following bounds are ti ght: 1 l k.
3. Knowing only the distributi on of the sensiti ve att ribute values; What bounds can you derive for l in
disti nct l-diversity?
Let L be the number of diff erent sensiti ve att ribute values. Then, there can also be at most L diff erent
values within each equivalence class w.r.t. to an Quasi-Identi fi er. Thus, l L.
Additi onal informati on: This bound is independent of the bound from (ii), as the former one operates
only on the Quasi-Identi fi er columns and this one solely considers the sensiti ve att ribute.
\end{document}
